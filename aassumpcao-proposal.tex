\documentclass[]{article}
\usepackage{lmodern}
\usepackage{amssymb,amsmath}
\usepackage{ifxetex,ifluatex}
\usepackage{fixltx2e} % provides \textsubscript
\ifnum 0\ifxetex 1\fi\ifluatex 1\fi=0 % if pdftex
  \usepackage[T1]{fontenc}
  \usepackage[utf8]{inputenc}
\else % if luatex or xelatex
  \ifxetex
    \usepackage{mathspec}
  \else
    \usepackage{fontspec}
  \fi
  \defaultfontfeatures{Ligatures=TeX,Scale=MatchLowercase}
\fi
% use upquote if available, for straight quotes in verbatim environments
\IfFileExists{upquote.sty}{\usepackage{upquote}}{}
% use microtype if available
\IfFileExists{microtype.sty}{%
\usepackage{microtype}
\UseMicrotypeSet[protrusion]{basicmath} % disable protrusion for tt fonts
}{}
\usepackage[margin=1in]{geometry}
\usepackage{hyperref}
\hypersetup{unicode=true,
            pdfborder={0 0 0},
            breaklinks=true}
\urlstyle{same}  % don't use monospace font for urls
\usepackage{longtable,booktabs}
\usepackage{graphicx,grffile}
\makeatletter
\def\maxwidth{\ifdim\Gin@nat@width>\linewidth\linewidth\else\Gin@nat@width\fi}
\def\maxheight{\ifdim\Gin@nat@height>\textheight\textheight\else\Gin@nat@height\fi}
\makeatother
% Scale images if necessary, so that they will not overflow the page
% margins by default, and it is still possible to overwrite the defaults
% using explicit options in \includegraphics[width, height, ...]{}
\setkeys{Gin}{width=\maxwidth,height=\maxheight,keepaspectratio}
\IfFileExists{parskip.sty}{%
\usepackage{parskip}
}{% else
\setlength{\parindent}{0pt}
\setlength{\parskip}{6pt plus 2pt minus 1pt}
}
\setlength{\emergencystretch}{3em}  % prevent overfull lines
\providecommand{\tightlist}{%
  \setlength{\itemsep}{0pt}\setlength{\parskip}{0pt}}
\setcounter{secnumdepth}{5}
% Redefines (sub)paragraphs to behave more like sections
\ifx\paragraph\undefined\else
\let\oldparagraph\paragraph
\renewcommand{\paragraph}[1]{\oldparagraph{#1}\mbox{}}
\fi
\ifx\subparagraph\undefined\else
\let\oldsubparagraph\subparagraph
\renewcommand{\subparagraph}[1]{\oldsubparagraph{#1}\mbox{}}
\fi

%%% Use protect on footnotes to avoid problems with footnotes in titles
\let\rmarkdownfootnote\footnote%
\def\footnote{\protect\rmarkdownfootnote}

%%% Change title format to be more compact
\usepackage{titling}

% Create subtitle command for use in maketitle
\newcommand{\subtitle}[1]{
  \posttitle{
    \begin{center}\large#1\end{center}
    }
}

\setlength{\droptitle}{-2em}

  \title{}
    \pretitle{\vspace{\droptitle}}
  \posttitle{}
    \author{}
    \preauthor{}\postauthor{}
    \date{}
    \predate{}\postdate{}
  
\usepackage{float}
\usepackage{tikz}
\usepackage{setspace}
\onehalfspacing

\begin{document}

\hypertarget{three-essays-on-sanctions-of-politicians-in-brazil}{%
\section*{Three Essays on Sanctions of Politicians in
Brazil}\label{three-essays-on-sanctions-of-politicians-in-brazil}}
\addcontentsline{toc}{section}{Three Essays on Sanctions of Politicians
in Brazil}

\emph{Andre Assumpcao}

\hypertarget{summary}{%
\subsection*{Summary}\label{summary}}
\addcontentsline{toc}{subsection}{Summary}

This dissertation project investigates the relationship between legal
sanctions and politics in Brazil. In the first paper, I look at the
effect of convictions for electoral infractions on electoral performance
in four municipal elections between 2004 and 2016. The second paper
tests whether State Court judges significantly rule in favor of
politicians involved in small court claims. Finally, the last paper
investigates whether active and passive transparency simultaneously
improve government performance and increase the number of legal
sanctions for government wrongdoing. Besides novel research questions,
this project also uses new data sources and innovative research methods
to advance the literature tying law, political science, and economics.

\hypertarget{title1}{%
\section{Electoral Crimes under Democratic Rule: Evidence from
Brazil}\label{title1}}

\hypertarget{summary-1}{%
\subsection{Summary}\label{summary-1}}

Office-seeking politicians employ various tactics to get elected,
including actions banned by electoral oversight agencies. Using a unique
dataset on candidate eligibility constructed from judicial decisions of
the Brazilian Electoral Court (TSE), I estimate the impact of electoral
code violations on performance at municipal elections between 2004-2016.
I recover causal effects of violations using a unique feature from
Brazil's electoral system in which candidates run for office, and can be
voted on election day, even if the Electoral Court has not issued a
final ruling on whether their candidacies met all code requirements.

\hypertarget{main-research-question}{%
\subsection{Main Research Question}\label{main-research-question}}

Do politicians who violate electoral code perform better at the ballot
than those who do not?

\hypertarget{hypotheses}{%
\subsection{Hypotheses}\label{hypotheses}}

\begin{enumerate}
\item
  Candidates who are cleared of electoral violations will perform better
  at municipal elections than candidates who are found guilty.
\item
  Conviction-free candidates will performance proportionately better
  when there are more convicted politicians in the same race.
\item
  The electoral punishment for electoral infractions is increasing in
  their severity (e.g.~corruption, abuse of power, vote buying,
  unauthorized marketing, etc).
\end{enumerate}

\hypertarget{outcomes}{%
\subsection{Outcomes}\label{outcomes}}

\begin{enumerate}
\item
  Whether candidate was the most voted for in mayor elections
  (majoritarian system) or the last most voted for in city council
  elections (proportional system) -- equivalent to being elected had
  they all been cleared of accusations.
\item
  Vote share.
\item
  Vote distance to elected candidates.
\end{enumerate}

\hypertarget{identification-strategy}{%
\subsection{Identification Strategy}\label{identification-strategy}}

IV where the instrumented (explanatory) variable is the candidacy ruling
issued by the Electoral Court before the elections (trial stage) and the
instrument is the decision issued after the elections (appeals stage)
for all politicians whose candidacy authorization had been appealed but
had not yet been tried by the appellate panel on election day.

\hypertarget{data}{%
\subsection{Data}\label{data}}

Election results, candidate, judge, and municipal controls for municipal
elections in Brazil between 2004-2016. All available at various
Electoral Court (TSE) websites and the National Statistics Office
(IBGE).

\hypertarget{contribution-to-literature}{%
\subsection{Contribution to
Literature}\label{contribution-to-literature}}

It is the first paper to measure the electoral punishment for electoral
crimes under democratic regimes; it implements a new identification
strategy using different stages of the judicial process; it uses a novel
database of primary data on judicial rulings from electoral oversight
authorities.

\hypertarget{title2}{%
\section{Judicial Favoritism of Politicians: Evidence from Small Claim
Courts}\label{title2}}

\hypertarget{summary-2}{%
\subsection{Summary}\label{summary-2}}

Judicial favoritism has long been a subject of research in law,
economics, and political science. However, scholars have mainly focused
on gender and ethnicity bias but have largely ignored whether judges
treat politicians in the same way as ordinary citizens. I use a unique
dataset of judicial decisions in small claims courts in the state of São
Paulo, Brazil, where cases are assigned to judges at random, to verify
whether local politicians have a higher winning rate against other
plaintiffs or defendants. Under judicial independence, politicians
should have the same win rate at trial than any other citizen.

\hypertarget{main-research-question-1}{%
\subsection{Main Research Question}\label{main-research-question-1}}

Are politicians more likely to receive favorable rulings in small claim
courts?

\hypertarget{hypotheses-1}{%
\subsection{Hypotheses}\label{hypotheses-1}}

\begin{enumerate}
\item
  Politicians have a higher winning rate at the trial stage in small
  court claims against their counterparts.
\item
  In cases where claims are of higher monetary value, politicians should
  see an even larger winning rate than at cases of lower value.
\item
  Proximity to elections increases the winning rates for politicians on
  the campaign trail.
\end{enumerate}

\hypertarget{outcomes-1}{%
\subsection{Outcomes}\label{outcomes-1}}

\begin{enumerate}
\item
  Whether politicians have had the case ruled in their favor.
\item
  The amount awarded to (or avoided by) politicians in small claim court
  cases.
\end{enumerate}

\hypertarget{identification-strategy-1}{%
\subsection{Identification Strategy}\label{identification-strategy-1}}

Natural Experiment. State Courts assign cases at random when the
judicial district has more than one judge on the bench. I combine
empirical strategies in Shayo \& Zussman
(\protect\hyperlink{ref-ShayoJudicialIngroupBias2011}{2011}), Abrams,
Bertrand, \& Mullainathan
(\protect\hyperlink{ref-AbramsJudgesVaryTheir2012}{2012}), and
Sanchez-Martinez
(\protect\hyperlink{ref-Sanchez-MartinezDismantlingInstitutionsCourt2018}{2018})
to test random assignment and provide robustness checks against
potential spurious relationships between being a politician and having a
favorable court outcome.

\hypertarget{data-1}{%
\subsection{Data}\label{data-1}}

São Paulo State Court (TJ-SP) rulings involving candidates running for
office in the State of São Paulo between 2010-2016 electoral cycles.
Judicial district, judge and politicians' individual characteristics
from the TJ-SP, Electoral Court (TSE), and the National Statistics
Office (IBGE).

\hypertarget{contribution-to-literature-1}{%
\subsection{Contribution to
Literature}\label{contribution-to-literature-1}}

It is amongst the new, few papers to investigate judicial bias for
individual politicians and it contributes to the literature on the
benefits of political connectedness.

\hypertarget{title3}{%
\section{Active and Passive Transparency in Brazilian
Municipalities}\label{title3}}

\hypertarget{summary-3}{%
\subsection{Summary}\label{summary-3}}

An important part of government accountability is the obligation of
public officials to inform and explain their actions (Bovens,
\protect\hyperlink{ref-BovensAnalysingAssessingAccountability2007}{2007};
Schedler,
\protect\hyperlink{ref-SchedlerConceptualizingAccountability2012}{2012}).
In this paper, I propose and analyze two related forms of government
accountability: \emph{active transparency}, in which government actively
reveals policy information via intra-government auditing and monitoring,
and \emph{passive transparency}, in which government passively reveals
information through freedom of information (FOIA) requests. Using a
natural two-by-two factorial experiment design in Brazilian
municipalities between 2006 and 2017, I measure the effects of active
and passive transparency on government performance, sanctions,
corruption, and transparency outcomes.

\hypertarget{main-research-question-2}{%
\subsection{Main Research Question}\label{main-research-question-2}}

Does passive transparency contribute anything else beyond active
transparency in improving government performance and increasing the
number of sanctions applied for government wrongdoing?

\hypertarget{hypotheses-2}{%
\subsection{Hypotheses}\label{hypotheses-2}}

\begin{enumerate}
\item
  \emph{Active transparency} measures unconditionally improve
  performance and increase the number of individual and company-wide
  sanctions.
\item
  \emph{Passive transparency} only marginally improves performance and
  increases sanctions when \emph{active transparency} policies are in
  place.
\item
  In the absence of \emph{active transparency} measures, \emph{passive
  transparency} has no effect on improving performance and does not
  increase the number of sanctions for individuals and companies found
  guilty of any wrongdoing.
\end{enumerate}

\hypertarget{outcomes-2}{%
\subsection{Outcomes}\label{outcomes-2}}

\begin{enumerate}
\item
  Performance (across all groups):

  \begin{enumerate}
  \item
    number of online or in-person services available to the public.
  \item
    the existence of municipal development plan.
  \end{enumerate}
\item
  Sanctions (across all groups):

  \begin{enumerate}
  \item
    whether the municipality had any public official convicted/fired for
    wrongdoing.
  \item
    whether local companies have been entered into blacklist of
    government providers.
  \item
    whether the municipality was targeted by Federal Police in
    corruption crackdowns.
  \end{enumerate}
\item
  Corruption (across passive transparency groups):

  \begin{enumerate}
  \item
    corruption findings over total investigations.
  \item
    amount potentially lost to corruption over the total amount
    investigated.
  \end{enumerate}
\item
  Transparency (across active transparency groups):

  \begin{enumerate}
  \item
    whether the municipality responded in time to four FOIA requests.
  \item
    whether the municipality provided correct answers to four FOIA
    requests.
  \end{enumerate}
\end{enumerate}

\hypertarget{identification-strategy-2}{%
\subsection{Identification Strategy}\label{identification-strategy-2}}

Natural experiment coming from the combination of two simultaneous
exogenous shocks: randomized audits (active transparency) plus the
nationwide implementation of the freedom of information act in 2012
(passive transparency). Municipalities fall into one of three treatments
or one control group: audit after FOIA (active and passive
transparency), audit before FOIA (active transparency), non-audit after
FOIA (passive transparency), and non-audit before FOIA (control).

\hypertarget{data-2}{%
\subsection{Data}\label{data-2}}

Socioeconomic factors and policy outcomes from the National Statistics
Office (IBGE); Random audits and transparency measures from two programs
run by the Office of the Comptroller-General (CGU); Sanctions for
individuals and companies and crackdowns from CGU; Convictions from the
National Council of Justice (CNJ).

\hypertarget{contribution-and-literature}{%
\subsection{Contribution and
Literature}\label{contribution-and-literature}}

First paper providing disaggregated evidence for the effect of passive
transparency (FOIA) in development settings; paper advances theory by
breaking transparency into active and passive arms; new transparency
dataset and innovative research design.

\hypertarget{references}{%
\subsection*{References}\label{references}}
\addcontentsline{toc}{subsection}{References}

\hypertarget{refs}{}
\leavevmode\hypertarget{ref-AbramsJudgesVaryTheir2012}{}%
Abrams, D. S., Bertrand, M., \& Mullainathan, S. (2012). Do Judges Vary
in Their Treatment of Race? \emph{The Journal of Legal Studies},
\emph{41}(2), 347--383.

\leavevmode\hypertarget{ref-BovensAnalysingAssessingAccountability2007}{}%
Bovens, M. (2007). Analysing and Assessing Accountability: A Conceptual
Framework. \emph{European Law Journal}, \emph{13}(4), 447--468.

\leavevmode\hypertarget{ref-CallenInstitutionalCorruptionElection2015}{}%
Callen, M., \& Long, J. D. (2015). Institutional Corruption and Election
Fraud: Evidence from a Field Experiment in Afghanistan. \emph{American
Economic Review}, \emph{105}(1), 354--381.

\leavevmode\hypertarget{ref-CarrerasTrustElectionsVote2013}{}%
Carreras, M., \& İrepoğlu, Y. (2013). Trust in Elections, Vote Buying,
and Turnout in Latin America. \emph{Electoral Studies}, \emph{32}(4),
609--619.

\leavevmode\hypertarget{ref-CordisSunshineDisinfectantEffect2014}{}%
Cordis, A. S., \& Warren, P. L. (2014). Sunshine as Disinfectant: The
Effect of State Freedom of Information Act Laws on Public Corruption.
\emph{Journal of Public Economics}, \emph{115}, 18--36.

\leavevmode\hypertarget{ref-DonnoDoesCheatingPay2012}{}%
Donno, D., \& Roussias, N. (2012). Does Cheating Pay? The Effect of
Electoral Misconduct on Party Systems. \emph{Comparative Political
Studies}, \emph{45}(5), 575--605.

\leavevmode\hypertarget{ref-Fernandez-VazquezRootingOutCorruption2016}{}%
Fernández-Vázquez, P., Barberá, P., \& Rivero, G. (2016). Rooting Out
Corruption or Rooting for Corruption? The Heterogeneous Electoral
Consequences of Scandals. \emph{Political Science Research and Methods},
\emph{4}(2), 379--397.

\leavevmode\hypertarget{ref-IchinoDeterringDisplacingElectoral2012}{}%
Ichino, N., \& Schündeln, M. (2012). Deterring or Displacing Electoral
Irregularities? Spillover Effects of Observers in a Randomized Field
Experiment in Ghana. \emph{The Journal of Politics}, \emph{74}(1),
292--307.

\leavevmode\hypertarget{ref-KuoIllicitTacticsSubstitutes2017}{}%
Kuo, D., \& Teorell, J. (2017). Illicit Tactics as Substitutes: Election
Fraud, Ballot Reform, and Contested Congressional Elections in the
United States, 1860-1930. \emph{Comparative Political Studies},
\emph{50}(5), 665--696.

\leavevmode\hypertarget{ref-LuPoliticalConnectednessCourt2015}{}%
Lu, H., Pan, H., \& Zhang, C. (2015). Political Connectedness and Court
Outcomes: Evidence from Chinese Corporate Lawsuits. \emph{The Journal of
Law and Economics}, \emph{58}(4), 829--861.

\leavevmode\hypertarget{ref-MeijerTransparency2014}{}%
Meijer, A. (2014). Transparency. \emph{The Oxford Handbook of Public
Accountability}.

\leavevmode\hypertarget{ref-MichenerFOILawsWorld2011}{}%
Michener, G. (2011). FOI Laws Around the World. \emph{Journal of
Democracy}, \emph{22}(2), 145--159.

\leavevmode\hypertarget{ref-PeisakhinTransparencyCorruptionEvidence2012}{}%
Peisakhin, L. (2012). Transparency and Corruption: Evidence from India.
\emph{The Journal of Law and Economics}, \emph{55}(1), 129--149.

\leavevmode\hypertarget{ref-PeisakhinTransparencyEffectiveAnticorruption2010}{}%
Peisakhin, L., \& Pinto, P. (2010). Is Transparency an Effective
Anti-corruption Strategy? Evidence from a Field Experiment in India.
\emph{Regulation \& Governance}, \emph{4}(3), 261--280.

\leavevmode\hypertarget{ref-PratWrongKindTransparency2005}{}%
Prat, A. (2005). The Wrong Kind of Transparency. \emph{American Economic
Review}, \emph{95}(3), 862--877.

\leavevmode\hypertarget{ref-RoussiasTyingIncumbentsHands2018}{}%
Roussias, N., \& Ruiz-Rufino, R. (2018). ``Tying Incumbents' Hands'':
The Effects of Election Monitoring on Electoral Outcomes.
\emph{Electoral Studies}, \emph{54}, 116--127.

\leavevmode\hypertarget{ref-Sanchez-MartinezDismantlingInstitutionsCourt2018}{}%
Sanchez-Martinez, C. (2018). \emph{Dismantling Institutions: Court
Politicization and Discrimination in Public Employment Lawsuits} (PhD
Thesis). Stanford University.

\leavevmode\hypertarget{ref-SchedlerConceptualizingAccountability2012}{}%
Schedler, A. (2012). Conceptualizing Accountability. In \emph{The
Self-Restraining State: Power and Accountability in New Democracies}.

\leavevmode\hypertarget{ref-ShayoJudicialIngroupBias2011}{}%
Shayo, M., \& Zussman, A. (2011). Judicial Ingroup Bias in the Shadow of
Terrorism. \emph{The Quarterly Journal of Economics}, \emph{126}(3),
1447--1484.

\leavevmode\hypertarget{ref-VadlamannatiFreedomInformationLaws2016}{}%
Vadlamannati, K. C., \& Cooray, A. (2016). Do Freedom of Information
Laws Improve Bureaucratic Efficiency? An Empirical Investigation.
\emph{Oxford Economic Papers}, \emph{68}(4), 968--993.


\end{document}
