\documentclass[11pt]{article}
\usepackage{amssymb}
\usepackage{amsmath}
\usepackage{centernot}
\usepackage{amsfonts}
\usepackage{eurosym}
\usepackage{geometry}
\usepackage{ulem}
\usepackage{graphicx}
\usepackage{tikz}
\usepackage{rotating}
\usepackage{caption}
\usepackage{color}
\usepackage{setspace}
\usepackage{sectsty}
\usepackage{comment}
\usepackage{footmisc}
\usepackage[inline]{enumitem}
\usepackage{caption}
\usepackage{natbib}
\usepackage{pdflscape}
\usepackage{subfigure}
\usepackage{array}
\usepackage{titling}
\usepackage{multirow}
\usepackage{diagbox}
\usepackage{dcolumn}
\usepackage{makecell}
\usepackage[hidelinks]{hyperref}
\hypersetup{unicode = true}

\normalem

\onehalfspacing
\newtheorem{theorem}{Theorem}
\newtheorem{corollary}[theorem]{Corollary}
\newtheorem{proposition}{Proposition}
\newenvironment{proof}[1][Proof]{\noindent\textbf{#1.}}{\ \rule{0.5em}{0.5em}}

\newtheorem{hyp}{Hypothesis}
\newtheorem{subhyp}{Hypothesis}[hyp]
\renewcommand{\thesubhyp}{\thehyp\alph{subhyp}}
% \renewcommand{\labelenumi}{H.\arabic{enumi}.} % Redefine new labels for hyp
\newcommand{\T}{\rule{0pt}{2.6ex}}            % Top strut
\newcommand{\B}{\rule[-1.2ex]{0pt}{0pt}}      % Bottom strut
\newcommand{\red}[1]{{\color{red} #1}}
\newcommand{\blue}[1]{{\color{blue} #1}}
\newcommand{\ci}{\perp\!\!\!\perp}
\newcommand{\nci}{\centernot{\ci}}
\newcommand{\subtitle}[1]{\posttitle{\par\end{center}\begin{center}\large#1\end{center}\vskip0.5em}}
\newcommand{\refp}[1]{(\ref{#1})}

\newcolumntype{L}[1]{>{\raggedright\let\newline\\arraybackslash\hspace{0pt}}m{#1}}
\newcolumntype{C}[1]{>{\centering\let\newline\\arraybackslash\hspace{0pt}}m{#1}}
\newcolumntype{R}[1]{>{\raggedleft\let\newline\\arraybackslash\hspace{0pt}}m{#1}}

\geometry{left=1.0in,right=1.0in,top=1.0in,bottom=1.0in}

\begin{document}

\begin{titlepage}
\title{Three Essays on Sanctions of Politicians in Brazil}
\subtitle{A proposal submitted in partial fulfillment of the requirements for the Degree of Doctor of Philosophy in Public Policy}
\author{Andre Assumpcao\thanks{PhD Student, Department of Public Policy, The University of North Carolina at Chapel Hill. Contact details: \href{mailto:aassumpcao@unc.edu}{\textcolor{blue}{aassumpcao@unc.edu}}}}

\maketitle

\begin{abstract}
\noindent This dissertation project will investigate the relationship between legal sanctions and politics in Brazil. In the first paper, I look at the effect of convictions for electoral infractions on electoral performance in four municipal elections between 2004 and 2016. The second paper tests whether State Court judges significantly rule in favor of politicians involved in small claim court cases. Finally, the last paper investigates whether active and passive transparency simultaneously improve government performance and increase the number of legal sanctions for government wrongdoing. These papers contribute significantly to the literature in political science, economics, and law by exploring the relationship between legal sanctions and local political dynamics in developing countries. In addition, I also contribute new data sources in the form of judicial decisions and innovative identification strategies using institutional features of Brazilian electoral and judicial systems. \\
\vspace{0in} \\
\noindent\textbf{Keywords:} political economy of development; electoral politics; judicial politics; transparency; economics of crime. \\
\vspace{0in}
\bigskip

\end{abstract}

\setcounter{page}{0}

\thispagestyle{empty}

\end{titlepage}

\clearpage

% \section*{Summary} \label{sec:summary}

% \clearpage

\section{Electoral Crime Under Democracy: Evidence from Brazil} \label{sec:paper1}

\subsection{Introduction} \label{subsec:introduction_paper1}

In democratic regimes, office-seeking politicians employ various tactics to get elected. They might promise more resources to increase the provision of local public goods, such as schools, hospitals, or roads; they can promote their candidacies by running ads on TV and, more recently, on social media; they meet with their constituents and ask for their vote based on their personal connection. While these tactics are different, sometimes complementary means an individual could deploy to win an election, they all characterize \emph{play-by-the-rules} strategies, in which people follow legal provisions when running for elected office. Governments allow such electoral practices because they make electoral systems more competitive and increase access to political office to more groups representing its citizens. In this paper, however, I focus on less understood, ways to win an election -- when candidates \emph{break the rules} and deploy illegal tactics to shape results.

Scholars have not ignored these mechanisms used to win elections. \citet{LehoucqElectoralFraudCauses2003} offers a comprehensive account of electoral fraud, which takes up a variety of forms, such as procedural rule-breaking, illegal campaigning, violence, and even unequivocal vote buying. In a more recent study, \citet{Gans-MorseVarietiesClientelismMachine2013a} design a theoretical framework encompassing four types of clientelism practices (vote, turnout, and abstention buying, and double persuasion) and their adoption under five different institutional designs. They argue that the choice of illegal action is conditional on the design of electoral systems. For instance, in an environment of increased political polarization, we should expect to see more of turnout buying but less of vote buying.

Indeed, most studies looking into illegal electoral tactics have two common characteristics: first, they are largely concerned with coercive threats that prevent free and fair elections, as suggested by \citet{MaresBuyingExpropriatingStealing2016}; second, they focus heavily on non or partially democratic regimes, evidenced by the vast literature on electoral authoritarianism \citep{LevitskyRiseCompetitiveAuthoritarianism2002,GandhiElectionsAuthoritarianism2009,IchinoDeterringDisplacingElectoral2012,SchedlerElectoralAuthoritarianism2015,AsunkaElectoralFraudViolence2017a}. This is a rich literature that helps understanding the use of elections for regime consolidation and continuity. Nevertheless, I address two unexplored issues that are supplemental to the established literature investigating electoral fraud.

The first contribution I advance is uncovering the effect of electoral infractions that are harder to detect or whose relationship with electoral outcomes is less known or well understood. For instance, politicians might use illegal forms of advertising or slush funds to spend beyond their campaign limits in order to win an election. Likewise, candidates and political parties might put forth candidacies for public office even if they do not meet all electoral requirements, a particular feature of Brazilian electoral law,\footnote{For instance, parties need to file financial records proving they are financially solid and candidates must not have been convicted of crimes at the appellate level either at the State and Federal judicial systems.} as a means of raising their profiles for future elections. These strategic moves are less easily understood than flat out vote buying, for example. The second contribution is precisely understanding how electoral infractions shape electoral outcomes in large democracies. Brazil is one of the top five largest democracies in the world and as such is an important research setting for understanding the use of illegal electoral tactics. Beyond just size, and despite a recent fallback, the quality of Brazil's democracy makes it an important case study: since 2006, the country consistently ranks in the top 20\% countries in the V-Dem Electoral Democracy Index \citep{CoppedgeVDemCountryYearDataset2018}.

Besides the electoral fraud scholarship, the present study contributes to the broader literature of political economy of development. Brazil has an unique institutional design in which the judiciary branch has an entire system of State (TRE) and Federal (TSE) electoral courts resolving electoral claims. Their mandate is to guarantee free and fair competition for public office, enforcing the Brazilian Electoral Code of 1965 and subsequent legislation, and to prevent that candidates not meeting code requirements join electoral races. To the extent that the Electoral Courts are successful in rooting out this type of wrongdoing, we should expect more electoral accountability from office-holding politicians. They would avoid illegal tactics to preserve their future career prospects. Understanding if electoral systems as such are effective should provide an important takeaway for countries sharing the same institutional design. In addition, this paper investigates another source of judiciary power beyond settling legal disputes between economic agents; since every political candidate in Brazil needs a judicial authorization to run for office, the Electoral Court holds an enormous amount of power in shaping up political representation -- an unusual role played by more judiciaries.

Another important contribution in this study is the use of court documents as data. I collect and code judicial rulings from TRE and TSE courts on candidacies for municipal office in Brazilian elections between 2004 and 2016. For a subset of these documents, I extract information on individual judge characteristics and electoral irregularities candidates are accused of, and ultimately convicted for, when running for office. Not only can I provide an estimate for the average treatment effect of electoral infractions on ballot performance but I can also estimate heterogeneous effects by type of violation for two of these elections, 2012 and 2016. This project forms part of a recent wave of studies using court documents to measure economic and political outcomes in development settings \citep{Sanchez-MartinezDismantlingInstitutionsCourt2018,LambaisJudicialSubversionEvidence2018}.

Using these court documents, I recover the causal effect of electoral irregularities adopting an instrumental variables (IV) strategy. Since the judicial ruling on candidacies is composed of up to three instances of review, I can instrument decisions at the trial stage (first instance) for appellate decisions (second or third instance) for a subset of candidates who have untried appeals by election day. In such cases, the electoral code in Brazil establishes that candidates can be voted for, and that their votes should be counted, regardless of the infraction brought against their candidacy in the first place. While the trial ruling is endogenous, e.g.~potentially correlated with other factors determining electoral prospects, appellate rulings issued \emph{after} election day cannot influence electoral outcomes beyond their connection with trial decisions. Thus, for a subsample of candidates running for office who have an untried appeal standing at the time of election, I can identify the causal effect of electoral violations on performance.

The preliminary results are encouraging. First-stage \emph{F}-statistics are large and significant at 1 percent for both models including and excluding candidate characteristics ($F = 1{,}092$ and $F = 16{,}356$). OLS and reduced-form parameter estimates are very close in magnitude and significant at 1 percent, indicating that the instrument is orthogonal to the error term in the first stage. In my preferred IV model, including candidate characteristics as controls, having been convicted of an electoral crime reduces the probability of election and total vote share for mayor and city councilor candidates by 28.8 and 17.9 percentage points, respectively. Though we should be careful when comparing these results with studies looking at punishment for corruption in Brazil \citep{FerrazExposingCorruptPoliticians2008b,FerrazElectoralAccountabilityCorruption2011a,WintersLackingInformationCondoning2013}, which is a more severe crime prosecuted by various other legal authorities in Brazil, the evidence consistently points to the negative impact on electoral performance when politicians decide to break rules.

In the remainder of this proposal, I explain the institutional background allowing for causal identification in section \ref{subsec:background_paper1}, present the data in section \ref{subsec:data_paper1}, and discuss the theoretical mechanism underlying the relationship between electoral crimes and performance in section \ref{subsec:theory_paper1}. Section \ref{subsec:methods_paper1} discusses the empirical strategy and section \ref{subsec:results_paper1} presents preliminary results. Section \ref{subsec:conclusion_paper1} concludes laying out the additional work which will be carried out to fully develop this paper.

\subsection{Institutional Background} \label{subsec:background_paper1}

Brazilian Federal (TSE) and State Electoral Court (TRE) systems have existed intermittently since 1932 but only became institutionally relevant after the country's return to democracy in 1985. Since then, electoral courts have a fundamental role in guaranteeing free and fair elections. Their mandate is to enforce the Electoral Code of 1965 and subsequent legislation, particularly the law establishing conditions for ineligibility to public office (1990), the Law of Political Parties (1995), the Law of Elections (1997), and the Clean Slate Act of 2010.

These courts are tasked with four main functions: (i) electoral rule-making; (ii) judicial consultations clarifying and establishing jurisprudence for conflicting electoral norms; (iii) administration of the electoral process, which consists in publishing the electoral calendar, testing voting machines, distributing voting machines to all districts, counting and publishing electoral results, etc; and, finally, (iv) conflict resolution on claims of breach of electoral law.

In this project, I am mostly interested in the courts' conflict resolution function and its underlying judicial review process. According to Brazilian Law, every individual running for office, at every level, has to submit proper documentation proving that they meet eligibility requirements for the office they are running; for instance, they should be 35 years of age or older to run for president or senator; executive-office holders, if running for any other elected office, must step down from their current post six months before election day. Every electoral cycle, the highest-level electoral court, TSE, establishes a schedule for submission of all these documents, which are reviewed at lower-level courts by electoral judges who issue sentences authorizing every single candidacy in the country. This is the main institutional feature that allows for causal identification of electoral irregularities on performance.

An example helps illustrate this point. Municipal elections took place on October 2, 2016. The deadline for submitting all candidacy documents was August 15, 2016. Between August 15 and September 12, electoral courts review and authorize each candidacy for mayor or city councilor. The process starts at the electoral district in which the candidate is running for office, and their trial ruling comes out of the designated electoral judge for that district. These judges are part of the State Court system and, when appointed to the electoral bench, are on leave from their original tenured positions at the state system. They serve on two-year mandates, with one reappointment allowed, so they never oversee the same district for more than one electoral cycle. If either a candidate or someone else, such as opponents or the Office of Electoral Prosecutions (MPE), files an appeal to the trial ruling, the case is presented before a panel of three judges at the State Electoral Court TRE. There are seven appellate court justices in each state's TRE, serving up to four-year mandates, and they are immune to local politics. In any state, six of these judges are voted in by their fellow tenured judges at the State and Federal Court systems and the last member is appointed by the President of Brazil. If plaintiffs or defendants are unhappy with the appellate court decision, they can bring their case before with the Federal Court TSE, which serves as the third and final instance of judicial review for mayor and city councilor candidates.

The September 12\footnote{The exact day varies marginally every cycle. In 2018, for instance, deadline for candidacy submission was Aug 15, last day for loading candidate information was Sep 17, and election day was Oct 7.} date is the key institutional feature that allows observing performance for politicians who violate electoral rules. It is the last day for entering candidate information onto electronic voting (EV) machines distributed at every single polling station in the country.\footnote{\citet{FujiwaraVotingTechnologyPolitical2015} describes this technology in detail.} All candidates who have untried appeals by this date will have their information loaded, and thus can be voted for, in the EV machines on election day. For this reason, I can observe the electoral performance of candidates who eventually are convicted of electoral violations and compare to candidates who are cleared of the same accusations. If candidates have had a final ruling before September 12, or if they have decided not to appeal their trial sentence, I cannot observed their performance because TSE will not include their information in the EV machines.

Exogenous variation in convictions for electoral violation comes from the timing according to which appeals sentences are issued by higher-level courts. Often, the high number of candidates running for municipal office, the judicial backlog, or the conditions of a particular electoral race make it difficult for electoral courts to hand out final decisions by Sep 12. Moreover, since candidates with outstanding appeals will have their information loaded onto EV machines regardless, there is no strong incentive for courts to issue decisions between then and Oct 2. In the lead-up to election day, judges and court officials are working around the clock making sure that 540,000+ EV machines are delivered to 450,000+ voting stations across the whole territory of Brazil; judges are ruling on smaller electoral cases that might or might not be included in candidacy cases; court officials are meeting with political parties and discussing local electoral situations, etc. It is not uncommon, therefore, that final decisions are issued only after election day, specially in municipal elections, when stakes are lower than in federal elections.

When candidate appeals are not ruled in time for elections, candidate information (picture, name, voting number) is displayed in the EV machine but their votes are computed \emph{sub judice} -- their vote count will be considered valid only when the TRE or TSE publish their final decision on the candidacy. Effectively, thus, the decisions at the appeals stage cannot affect electoral outcomes, since they are issued only after election day has passed, but they bear a strong relationship to the sentence handed out by the trial judge in each electoral district. Decisions at trial are mostly endogenous to electoral outcomes, but the use of appeals as instruments leaves out only their exogenous part which allows for causal identification.

The main limitation of this study is that I can only recover causal effects of electoral violations under restrictive special conditions pertaining to municipal elections in Brazil. At any other electoral race, both the trial and appeals stages are handled by the TRE and this might shape the way electoral judges issue rulings in response to the importance of the office for which a candidacy has been presented. For instance, senators are much more influential than city councilors and have a direct channel of communication with the President of Brazil, who is responsible for appointing one judge per TRE. Second, there are a number of candidates who do not appeal their trial ruling and as such do not appear on the EV machine on election day. Thus, I cannot observe their ballot performance. Though it is likely that these candidates are heterogeneous in many dimensions when compared to candidates who have outstanding appeals, such as their political experience, or their drive to hold elected office, it would still be interesting if I could estimate causal effects for the entire distribution of electoral irregularities instead of the narrow sample at hand. Still, the lack of evidence of the effect of electoral violations on ballot performance for large democracies speaks to the relevance of this project.

\subsection{Data} \label{subsec:data_paper1}

The main data source for electoral performance is TSE's repository of electoral data. TSE publishes electoral results, vote counts, candidate's individual characteristics, and their candidacy's situation on election day for all elections since 1994. I focus on the municipal elections after the introduction of the EV machine in 2002 for even performance measures across elections and municipalities. My sample is composed of 9,469 candidates for mayor or city councilor who appealed, or had third-parties appealing, the trial sentence on their candidacy authorization. These candidates have been displayed in the EV device and could have been voted for on election day. Their candidacy remained pending after elections and they have only been allowed to take up office once a final ruling was issued. I create three outcome measures from TSE's data: (1) \emph{the probability of election}, which is a binary variable with value one when the candidate has received enough votes to be elected. For mayor candidates, under majoritarian rule, this means 50 percent plus one of all valid votes. For city councilor candidates, under proportional rule, this means having received enough votes to rank amongst the most voted candidates within the designated number of vacancies for each municipality; (2) \emph{the vote distance to elected candidate}, which is an extensive margin measure of the effect of electoral irregularities on performance, ranging from zero to 100; (3) \emph{the  vote share} as a percentage of total valid votes.

Data on the reasons for rejecting candidacies are scraped from the TSE website, which makes all their court documents public. I have developed a web scraper that downloads the case number for each municipal candidacy and uses these numbers to find case files and sentences. Though I can find case numbers for all candidates in my sample, I cannot find case information for elections in 2004 and 2008 due to limitations in the TSE website. Thus, I produce heterogeneous treatment effect estimates only for the more recent elections in 2012 and 2016. For the benefit of research transparency, all programs and data are freely available online.\footnote{I maintain a \href{https://www.github.com/aassumpcao/tse-scraper}{\textcolor{blue}{github}} repository storing all programs, data, and manuscripts used in this project. For the time being, this page is set to private and the latter link only provides access to the web scrapers.}

I have already collected all data on candidates and electoral performance from the TSE repository. Candidate characteristics are age, gender, political experience, education, marital status, and the reported amount of funds they have spent on the campaign trail. There are, however, a few data management and transformation tasks remaining. I have coded political experience as a binary variable for when candidates report holding office at the time of elections, but I have yet to include as experience candidates who have run for office multiple times and, when running for city councilor, candidates who belong to the strongest coalition at the time of elections. Second, I need to collect further data on campaign expenditures; I have coded reported spending using the information in TSE's candidacy dataset, which has many missing values, but the comprehensive data on funding belongs in TSE's campaign funds database.

As far as obtaining the data from court documents go, I have already written the web scrapers for obtaining case numbers and information, but have only downloaded case numbers. The remaining steps for completing the data collection task are downloading all possible case information in HTML format, one per case, and parsing HTML files into human-readable text. Next, I should use text analysis software to read these judicial decisions and code them into the appropriate bins for each electoral infraction. Though I am yet to do this for the whole sample, I have completed this entire process for a random sample of 1,000 candidates in the 2016 election as a trial run, so I am well versed on the necessary steps for doing the remaining judicial cases used in this paper.

\subsection{Theory} \label{subsec:theory_paper1}

Assume a representative candidate \emph{A} is running for office \emph{X}. They will adopt various electoral tactics in order to gather more votes on election day. For instance, they could devote more funds to ads on social media than on physical billboards in townships and highways. Any country's electoral authority lays out rules for office races and campaigns, such as the maximum amount of spending per candidate, means of advertising, dates during which campaigning is authorized, age limits, etc. In the hypothetical case above, \emph{A} is playing by the rules by deploying a \emph{legal} electoral tactic. In democratic countries, these tactics promote political competition, representation of various diverse demographics in the population, and make politicians more accountable to their constituents. In this setting, the representative candidate running for office maximizes the following utility function \refp{eq:1}: \\
\begin{equation} \label{eq:1}
  \text{max} \thickspace P(\text{election}) = f(X, \varepsilon)
\end{equation}

Their probability of election $P$ is a function of a series of observable characteristics represented by a matrix $X$, as described above, which can benefit or hurt a candidate's electoral chances. These factors could be anything form individual ability to marketing strategies. To the best of their ability, the representative candidate clearly wants to adopt beneficial tactics and avoid negative ones. The error term $\varepsilon$ captures any other non-observable or non-observed characteristic that simultaneously determines a candidate's chance to hold office; political ability is an example of the former, the deals candidates make with supporters or sponsors are the latter.

A fundamental factor present in the error term is any illegal action taken by candidates to win an election. In democratic regimes, candidates will hide their wrongdoing because of the negative press and ammunition this provides to their political opponents. They might be financing negative ads on their opponents, or paying journalists and agitators to disrupt their opponents campaign events, but do not want this to become public for fear of losing support. Also, if electoral authorities are serious about righting electoral wrongs, there will be even further incentives to hide electoral malfeasance. Therefore, illegal actions are not easily observed. Unfortunately, this is generally the case for social science studies trying to investigate the causes and consequences of crimes, which are by nature only partially observed phenomena. Ideally, we would like to observe the crime measure independently of the error term, such as in equation \refp{eq:2}:
\begin{equation} \label{eq:2}
  P(\text{election}) = f(X, c, \varepsilon)
\end{equation}
\begin{equation} \label{eq:3}
  c = f(r, p, \epsilon)
\end{equation}

Variable $c$ in equations \refp{eq:2} and \refp{eq:3} summarizes the illegal action taken by candidate \emph{A}. It is a function of electoral rewards $r$ and electoral or legal punishment $p$, summarized in equation \refp{eq:3}. The representative candidate \emph{A} compares $r$ and $p$ before deciding whether to adopt the forbidden tactic or not.

In addition to the difficulty of measuring crime variables, electoral outcomes are not generally observed in democratic regimes. In countries where electoral authorities are serious agencies, any candidate found guilty of severe electoral wrongdoing would be removed from the race; therefore, observing illegal action would not suffice for estimating effects of crime on the probability of election. We would lack the measurement of the outcome in the presence of severe crimes, which would bias the estimates toward low penalties in response to small crimes.

Fortunately, the institutional design of the municipal electoral process in Brazil makes it a perfect setting for estimating equation \refp{eq:2}, as it allows the simultaneous observation of both crimes and electoral outcomes under certain conditions. The dependent variable can be observed for candidates who have outstanding appeals by election day and we can compare a subsample of the entire candidate population for which crimes have been detected and are being prosecuted. The remaining problem for identifying the causal effect of electoral violations on performance is the relationship between $X$ and $c$. Candidates are likely to choose an illegal action based on the type and quality of political competition they are facing, their own political skills, and other unobserved factors. That said, I can claim with a fair amount of confidence that $cov(X,c) \neq 0$ and $cov(c, \varepsilon) \neq 0$. In section \ref{subsec:methods_paper1}, I explain how I address this problem and how I can recover the causal effects of crime for a subsample of municipal candidates in Brazil.

\subsection{Empirical Strategy} \label{subsec:methods_paper1}

In this paper, I adopt an instrumental variables approach that allows the causal identification of the effect of electoral irregularities on performance in Brazil. As described in section \ref{subsec:background_paper1}, I can only recover local average treatment effects for the subsample of candidates who are charged with electoral crimes, by the ruling judge or third-parties, and have an outstanding appeal on their trial decision. Candidates who break electoral code but are not detected are not part of this study, neither are candidates who have chosen not to appeal the sentence rejecting their candidacy for elected office. For this sample of candidates with untried appeals on election day, I estimate the following regression model in three ways and using three different measures of electoral performance:
\begin{equation} \label{eq:reg1}
  \begin{split}
    y_{i} = \alpha + \rho \cdot c_{i} + X\beta + \sum \lambda_{i, k} + \varepsilon_{i}
  \end{split}
\end{equation}

The dependent variable $y_{i}$ has three forms: (i) the probability of election, taking up value one when either the mayor or city councilor has had enough votes for election in their race; (ii) the vote distance to elected candidate(s), which is the percentage point margin between candidate $i$ vote share and that of the single elected candidate (when running for mayor) or last elected candidate (when running for city councilor); (iii) the total vote share of candidate $i$ received in their race. Using outcome one, I can measure the impact of the electoral violation on the most important outcome of any political campaign, i.e.~being elected; outcome two tells us about the benefit for committing crimes when candidates are trying to secure an electoral lead or narrow in on races in which they are trailing another candidate; the last outcome serves as a measure of the overall impact on candidate popularity if they are found guilty by the electoral court system.

The main independent variable is the binary indicator for convictions for electoral crime $c$ at the electoral court system for candidate $i$. If a candidacy has been rejected by the trial judge responsible for that electoral district, $c$ becomes one. I use convictions at trial in OLS regressions for benchmarking the effect on electoral performance; in reduced-form regressions, I replace convictions at trial for convictions on appeal -- which becomes one when the candidate has seen an unfavorable ruling at higher instances of the electoral court system. The reduced-form regressions hint at any potential correlation between instruments and outcomes beyond the channel via instrumented variables. I lastly estimate model \refp{eq:reg1} using two-stage least squares (2SLS) regressions, in which I instrument convictions at trial for convictions on appeal. Since I am looking at appellate court decisions issued after election day, the exclusion restriction is straightforward as the instrument is measured only \emph{after} outcomes have been observed. Any effect of appellate decisions only influences electoral performance via their relationship with convictions at trial. In such case, equation \refp{eq:reg1} and the first-stage are:
\begin{equation} \label{eq:reg2}
  \begin{split}
    y_{i} = \alpha + \rho \cdot \hat{c}_{i, \text{trial}} + X\beta + \sum \lambda_{i, k} + \varepsilon_{i}
  \end{split}
\end{equation}
\begin{equation} \label{eq:regfirststage}
  \begin{split}
    c_{i, \text{trial}} = \alpha + \rho \cdot c_{i, \text{appeals}} + X\beta + \sum \lambda_{i, k} + \varepsilon_{i}
  \end{split}
\end{equation}

For every specification of equations \refp{eq:reg2} and \refp{eq:regfirststage}, I estimate versions excluding and including individual characteristics (matrix $X$), such as candidate age, gender, marital and education status, political experience, and campaign spending. For the complete version of this paper, I also include a set of fixed-effects to capture any additional unobservable heterogeneity coming from year, election, and municipal conditions shared by subsets of candidates. Any remaining stochastic effect belongs the error term $\varepsilon$.

In addition to running these regression equations, I offer various other robustness checks confirming instrument strength and doing away with potential spurious relationship concerns. In the following section, I present first-stage results and the Hausman statistic to support my instrument choice. In the final version, I also include falsification tests using a sample of candidates who should not experience any electoral effect of judicial convictions on their candidacies.

\subsection{Preliminary Analysis} \label{subsec:results_paper1}

As a preview of the complete paper, I present in this section some preliminary results establishing the relationship between electoral violations and electoral performance. I have collected data on and used all electoral outcomes, candidate individual characteristics, and candidacy situations for estimating these preliminary regressions. Here I focus on the the local average treatment effects (LATE) of crimes regardless of the type of infraction for which candidates were found guilty or not guilty. Table \ref{tab:sumstats} reports descriptive statistics on the sample of candidates. \\


\begin{table}[!htbp] \centering
  \caption{Descriptive Statistics}
  \label{tab:sumstats}
\scriptsize
\begin{tabular}{@{\extracolsep{2pt}}lrrrrr}
\\[-1.8ex]\hline
\hline \\[-1.8ex]
& \multicolumn{1}{c}{N} & \multicolumn{1}{c}{Mean} & \multicolumn{1}{c}{St. Dev.} & \multicolumn{1}{c}{Min} & \multicolumn{1}{c}{Max} \\
\hline \\[-1.8ex]
Age                                           & 9,469 & 46.34 & 11.02 & 17 & 86 \\
Male                                          & 9,469 & .793 & .405 & 0 & 1 \\
Political Experience                          & 9,469 & .091 & .287 & 0 & 1 \\
Campaign Expenditures                         & 9,469 & 144,722 & 456,532 & 0 & 20,000,000 \\
Convicted at Trial                            & 9,469 & .641 & .480 & 0 & 1 \\
Convicted on Appeal                           & 9,469 & .537 & .499 & 0 & 1 \\
Probability of Election                       & 9,441 & .191 & .393 & 0.000 & 1 \\
Vote Distance to Elected Candidates (in p.p.) & 9,441 & -4.09 & 9.55 & -92.82 & 12.83 \\
Total Vote Share (in p.p.)                    & 9,441 & 10.131 & 17.983 & 0 & 100 \\
\hline
\hline \\[-1.8ex]
\end{tabular}
\end{table}


The average politician included in this study is 46 years-old and male (79 percent of the sample). About 10 percent of them were office-holders at the time of election and spent R\$ 144,722, on average, on the campaign trail (\$35,000 using the current exchange rate). Categorical variables are omitted from the table but the most frequent education level of politicians is high school (30 percent) and 62.5 percent of them report being married at the time of elections. Sixty-four percent of candidates have been convicted at trial, meaning their candidacies have not been authorized, while the remaining 36 percent have seen their opponents or the Office of Electoral Prosecutions (MPE) file an appeal against their candidacy with TRE. Approximately 53 percent have had a final ruling against their candidacy issued by the panel of appellate judges. \\

\begin{table}[!htbp]
  \caption{\label{tab:percentreversed}Percent of Candidacy Trial Rulings Reversed on Appeal}
  \centering
  \scriptsize
  \begin{tabular}{@{\extracolsep{12pt}}r|rr|r}
    \hline
    \hline
    & \multicolumn{2}{c}{\emph{Appeals}} \vline & Percent \T \B \\
    \multicolumn{1}{r}{\emph{Trial}} \vline & Affirmed & Reversed & Reversed \T \B \\
    \hline
    Not Convicted & 3379 & 22   & 0.6  \T \B \\
    Convicted     & 5059 & 1009 & 16.6 \T \B \\
    \hline
    \hline
  \end{tabular}
\end{table}

When I break down reversals by type of decision at trial, we can notice that favorable candidacy rulings are rarely reversed (0.6 percent); though the reversal of unfavorable rulings is more frequent (16.6 percent), overall we see that affirmed decisions are much more frequent than reversals (89.1 percent). This is the first anecdotal fact in support of instrument strength, since this is indicative of a strong unconditional correlation between decisions at both stages; nevertheless, I present stronger evidence in table \ref{tab:firststage}, where I tabulate the first-stage regressions of convictions at trial and on appeal. \\


\begin{table}[!htbp] \centering 
  \caption{First Stage Regressions of Convictions at Trial and on Appeal} 
  \label{tab:firststage} 
\scriptsize 
\begin{tabular}{@{\extracolsep{5pt}}lD{.}{.}{-3} D{.}{.}{-3} } 
\\[-1.8ex]\hline 
\hline \\[-1.8ex] 
 & \multicolumn{2}{c}{Outcome: Convicted on Appeal} \\ 
\cline{2-3} 
 & \multicolumn{1}{c}{First-Stage} & \multicolumn{1}{c}{First-Stage} \\ 
\\[-1.8ex] & \multicolumn{1}{c}{(1)} & \multicolumn{1}{c}{(2)}\\ 
\hline \\[-1.8ex] 
 Convicted on Appeal & .766^{***} & .757^{***} \\ 
  & (.006) & (.007) \\ 
  & & \\ 
\hline \\[-1.8ex] 
Individual Controls & \multicolumn{1}{c}{-} & \multicolumn{1}{c}{-} \\ 
\hline \\[-1.8ex] 
Observations & \multicolumn{1}{c}{9,469} & \multicolumn{1}{c}{9,469} \\ 
R$^{2}$ & \multicolumn{1}{c}{.633} & \multicolumn{1}{c}{.649} \\ 
Adjusted R$^{2}$ & \multicolumn{1}{c}{.633} & \multicolumn{1}{c}{.648} \\ 
Residual Std. Error & \multicolumn{1}{c}{.290 (df = 9467)} & \multicolumn{1}{c}{.285 (df = 9452)} \\ 
F Statistic & \multicolumn{1}{c}{16,359.970$^{***}$ (df = 1; 9467)} & \multicolumn{1}{c}{1,091.782$^{***}$ (df = 16; 9452)} \\ 
\hline 
\hline \\[-1.8ex] 
\textit{Note:}  & \multicolumn{2}{r}{$^{*}$p$<$0.1; $^{**}$p$<$0.05; $^{***}$p$<$0.01} \\ 
\end{tabular} 
\end{table} 


Not only the relationship between rulings at both stages are significant at 1 percent, but the magnitude of the parameters is large. Being convicted by a trial judge increases a politician's probability of an unfavorable ruling on appeal by approximately 76 percentage points, regardless of including individual control variables in the first-stage. In the presence of controls, the \emph{F}-statistic is significant at 1 percent and substantially larger ($F = 1{,}092$) than industry standards at $F = 10$. Adjusted-R\textsuperscript{2} is .648 and also lends support to the use of convictions on appeal as instruments for convictions at trial. In the complete version of this paper, I am going to discuss these results in more detail and in light of IV developments as suggested by \citet{ConleyPlausiblyExogenous2012}, \citet{ImbensInstrumentalVariablesEconometrician2014}, and \citet{AndrewsUnbiasedInstrumentalVariables2017}.

In table \ref{tab:hausman}, I perform another test of instrument strength using the Hausman statistic calculated from instrumental variables regressions (including covariates) for all three outcomes. In models one and three, we can safely reject the hypothesis that OLS and IV models are asymptotically equal (both are significant at 1 percent). For model two, I fail to reject the equality hypothesis, suggesting that the IV is not asymptotically different than OLS. Since the vote distance to elected candidates is substantially different if we are under majoritarian (mayor) or proportional (city council) elections, i.e.~it is smaller when votes are spread out across many more candidates, the weak instrument result is entirely plausible when estimating models with a single sample for both types of candidates. In the complete paper, I break down the sample for outcome two and show that the instrument is strong in each split sample case, again showing support for the identification strategy chosen for this project. \\

% latex table generated in R 3.5.1 by xtable 1.8-2 package
% Mon Nov 26 11:57:17 2018
\begin{table}[htbp]
\centering
\caption{Hausman Tests for Instrument Strength}
\label{tab:hausman}
\scriptsize
\begin{tabular}{lD{.}{.}{-2}D{.}{.}{-3}}
\\[-1.8ex] \hline
\hline \\[-1.8ex]
Outcome & \multicolumn{1}{r}{Hausman Statistic} & \multicolumn{1}{r}{\emph{p}-value} \\
\hline \\[-1.8ex]
Probability of Election             & 160.84 & 1.470$e-36$ \\
Vote Distance to Elected Candidates & 0.00   & .961 \\
Total Vote Share                    & 292.44 & 1.360$e-64$ \\
\\[-1.8ex] \hline
\hline \\[-1.8ex]
\end{tabular}
\end{table}


Tables \ref{tab:outcome1}-\ref{tab:outcome3} represent the bulk of the preliminary results in this proposal. Table \ref{tab:outcome1} reports the effect of electoral infractions on the probability of election of each candidate. For mayors, this variable turns on when the candidate was the most voted in their election. For city councilors, this variable turns on when the candidate has received enough votes to finish the election within the number of vacancies in their municipality. For instance, if municipality A has 12 seats in its city council, a candidate who received the same number, or more, votes than the 12\textsuperscript{th} placed candidate has outcome value one.\footnote{City council elections are not necessarily decided in such manner; TSE tallies up all votes in a single election and divides them up by the number of seats available. All candidates who have more votes than this mark are automatically elected to office; remaining seats go to the coalitions who have rounded up more votes. In most cases, however, votes are usually spread out across many candidates and coalitions, so being voted in as the last candidate within the number of available seats does guarantee their election and supports their coalitions to get further seats.} This is the most important outcome, and directly tests the theoretical claim suggested in section \ref{subsec:theory_paper1}.

\clearpage



\begin{table}[!htbp] \centering
  \caption{The Effect of Electoral Crimes on the Probability of Election}
  \label{tab:outcome1}
\scriptsize
\begin{tabular}{@{\extracolsep{-2pt}}lD{.}{.}{-3} D{.}{.}{-3} D{.}{.}{-3} D{.}{.}{-3} D{.}{.}{-3} D{.}{.}{-3} }
\\[-1.8ex]\hline
\hline \\[-1.8ex]
                     & \multicolumn{6}{c}{Outcome: Probability of Election} \\
\cline{2-7} \\ [-1.8ex]
                     & \multicolumn{1}{c}{OLS} & \multicolumn{1}{c}{OLS} & \multicolumn{1}{c}{Reduced-form} & \multicolumn{1}{c}{Reduced-form} & \multicolumn{1}{c}{IV} & \multicolumn{1}{c}{IV} \\ \\ [-1.8ex]
                     & \multicolumn{1}{c}{(1)} & \multicolumn{1}{c}{(2)} & \multicolumn{1}{c}{(3)} & \multicolumn{1}{c}{(4)} & \multicolumn{1}{c}{(5)} & \multicolumn{1}{c}{(6)}\\
\hline \\[-1.8ex]
 Convicted at Trial  & -.208^{***} & -.173^{***}  &             &             & -.272^{***} & -.288^{***} \\
                     & (.009)      & (.009)       &             &             & (.011)      & (.010) \\
                     &             &              &             &             &             & \\
 Convicted on Appeal &             &              & -.209^{***} & -.182^{***} &             &  \\
                     &             &              & (.008)      & (.008)      &             &  \\
                     &             &              &             &             &             & \\
\hline \\[-1.8ex]
Individual Controls & \multicolumn{1}{c}{-}              & \multicolumn{1}{c}{Yes}             & \multicolumn{1}{c}{-}              & \multicolumn{1}{c}{Yes}             & \multicolumn{1}{c}{-}           & \multicolumn{1}{c}{Yes} \\
\hline \\[-1.8ex]
Observations        & \multicolumn{1}{c}{9,441}          & \multicolumn{1}{c}{9,441}           & \multicolumn{1}{c}{9,441}          & \multicolumn{1}{c}{9,441}           & \multicolumn{1}{c}{9,441}       & \multicolumn{1}{c}{9,441} \\
R$^{2}$             & \multicolumn{1}{c}{.065}           & \multicolumn{1}{c}{.123}            & \multicolumn{1}{c}{.070}           & \multicolumn{1}{c}{.133}            & \multicolumn{1}{c}{.059}        & \multicolumn{1}{c}{.055} \\
Adjusted R$^{2}$    & \multicolumn{1}{c}{.065}           & \multicolumn{1}{c}{.122}            & \multicolumn{1}{c}{.070}           & \multicolumn{1}{c}{.131}            & \multicolumn{1}{c}{.058}        & \multicolumn{1}{c}{.055} \\
Residual Std. Error & \multicolumn{1}{c}{.380}           & \multicolumn{1}{c}{.368}            & \multicolumn{1}{c}{.379}           & \multicolumn{1}{c}{.366}            & \multicolumn{1}{c}{.381}        & \multicolumn{1}{c}{.382} \\
                    % & \multicolumn{1}{c}{(df = 9439)}    & \multicolumn{1}{c}{(df = 9424)}     & \multicolumn{1}{c}{(df = 9439)}    & \multicolumn{1}{c}{(df = 9424)}     & \multicolumn{1}{c}{(df = 9439)} & \multicolumn{1}{c}{(df = 9439)} \\
F-Statistic         & \multicolumn{1}{c}{652.4$^{***}$}  & \multicolumn{1}{c}{82.9$^{***}$}    & \multicolumn{1}{c}{715.4$^{***}$}  & \multicolumn{1}{c}{90.0$^{***}$}    & \multicolumn{1}{c}{-}           & \multicolumn{1}{c}{-} \\
                    & \multicolumn{1}{c}{(df = 1; 9439)} & \multicolumn{1}{c}{(df = 16; 9424)} & \multicolumn{1}{c}{(df = 1; 9439)} & \multicolumn{1}{c}{(df = 16; 9424)} &                                 & \\
\hline
\hline \\[-1.8ex]
\multicolumn{7}{l}{\textit{Note:} $^{*}$p$<$0.1; $^{**}$p$<$0.05; $^{***}$p$<$0.01.} \\
\end{tabular}
\end{table}


First, I draw attention to the effect of convictions on the probability of election in columns 1-4. At either the trial or appeals stage, being convicted for an electoral crime reduces the probability a candidate is elected by 20.8 and 20.9 percentage points in the absence of control variables and 17.3 and 18.2 percentage points in their presence. These are respectively the OLS and reduced-form regressions as described in section \ref{subsec:methods_paper1}. The conviction parameters are significant at 1 percent and very close in magnitude. In fact, the alignment between OLS and reduced-form estimates, including their adjusted-R\textsuperscript{2} and \emph{F}-statistics, is another source of support for the claim of independence across the instrument and the error term in the IV regression.

In columns 5-6, I report the IV estimates of electoral violations on probability of election. Being convicted for an electoral crime at trial has a negative impact on the election probability of 28.8 percentage points in the model including covariates (column 6). In line with corruption studies \citep{FerrazElectoralAccountabilityCorruption2011a,ChongLookingIncumbentExposing2013}, there is an electoral penalty for candidates who engage in illegal activities. For the complete version of this paper, I will include the fixed-effects discussed in section \ref{subsec:methods_paper1}, i.e.~year, election, and municipal indicators, and additional judge covariates.

In table \ref{tab:outcome2}, I look at the percentage point distance when convicted for electoral violations. The negative parameter estimates should be interpreted as the extensive margin effect of electoral irregularity; in other words, by how much they have lost their electoral race. The smaller the value, the less important the crime was as a factor for any candidate's electoral prospects.
\clearpage

\begin{table}[!htbp] \centering
  \caption{The Effect of Electoral Crimes on the Vote Distance to Elected Candidates}
  \label{tab:outcome2}
\scriptsize
\begin{tabular}{@{\extracolsep{-2pt}}lD{.}{.}{-3} D{.}{.}{-3} D{.}{.}{-3} D{.}{.}{-3} D{.}{.}{-3} D{.}{.}{-3} }
\\[-1.8ex]\hline
\hline \\[-1.8ex]
                     & \multicolumn{6}{c}{Outcome: Vote Distance to Elected Candidates (in p.p.)} \\
\cline{2-7} \\[-1.8ex]
                     & \multicolumn{1}{c}{OLS} & \multicolumn{1}{c}{OLS} & \multicolumn{1}{c}{Reduced-form} & \multicolumn{1}{c}{Reduced-form} & \multicolumn{1}{c}{IV} & \multicolumn{1}{c}{IV} \\
\\[-1.8ex]           & \multicolumn{1}{c}{(1)} & \multicolumn{1}{c}{(2)} & \multicolumn{1}{c}{(3)} & \multicolumn{1}{c}{(4)} & \multicolumn{1}{c}{(5)} & \multicolumn{1}{c}{(6)}\\
\hline \\[-1.8ex]
 Convicted at Trial  & -.308  & -.736^{***} &            &             & -.519^{**} & -.315 \\
                     & (.199) & (.206)      &            &             & (.254)     & (.251) \\
                     &        &             &            &             &            & \\
 Convicted on Appeal &        &             & -.399^{**} & -.751^{***} &            &  \\
                     &        &             & (.196)     & (.200)      &            &  \\
                     &        &             &            &             &            & \\
\hline \\[-1.8ex]
Individual Controls  & \multicolumn{1}{c}{-}              & \multicolumn{1}{c}{Yes}             & \multicolumn{1}{c}{-}              & \multicolumn{1}{c}{Yes}             & \multicolumn{1}{c}{-}     & \multicolumn{1}{c}{Yes} \\
\hline \\[-1.8ex]
Observations         & \multicolumn{1}{c}{9,441}          & \multicolumn{1}{c}{9,441}           & \multicolumn{1}{c}{9,441}          & \multicolumn{1}{c}{9,441}           & \multicolumn{1}{c}{9,441} & \multicolumn{1}{c}{9,441} \\
R$^{2}$              & \multicolumn{1}{c}{0.000}          & \multicolumn{1}{c}{.028}            & \multicolumn{1}{c}{0.000}          & \multicolumn{1}{c}{.028}            & \multicolumn{1}{c}{0.000} & \multicolumn{1}{c}{0.000} \\
Adjusted R$^{2}$     & \multicolumn{1}{c}{0.000}          & \multicolumn{1}{c}{.026}            & \multicolumn{1}{c}{0.000}          & \multicolumn{1}{c}{.026}            & \multicolumn{1}{c}{0.000} & \multicolumn{1}{c}{0.000} \\
Residual Std. Error  & \multicolumn{1}{c}{9.550}          & \multicolumn{1}{c}{9.426}           & \multicolumn{1}{c}{9.549}          & \multicolumn{1}{c}{9.425}           & \multicolumn{1}{c}{9.551} & \multicolumn{1}{c}{9.550} \\
F-Statistic          & \multicolumn{1}{c}{2.3}            & \multicolumn{1}{c}{16.7$^{***}$}    & \multicolumn{1}{c}{4.1$^{**}$}     & \multicolumn{1}{c}{16.9$^{***}$}    & \multicolumn{1}{c}{-}     & \multicolumn{1}{c}{-} \\
                     & \multicolumn{1}{c}{(df = 1; 9439)} & \multicolumn{1}{c}{(df = 16; 9424)} & \multicolumn{1}{c}{(df = 1; 9439)} & \multicolumn{1}{c}{(df = 16; 9424)} \\

\hline
\hline \\[-1.8ex]
\multicolumn{7}{l}{\textit{Note:} $^{*}$p$<$0.1; $^{**}$p$<$0.05; $^{***}$p$<$0.01} \\
\end{tabular}
\end{table}


The results in table \ref{tab:outcome2} are much less robust. There is a much larger disparity across the OLS and the reduced-form specifications, which are not similar in various dimensions: the magnitude of conviction parameters is much larger in the absence of covariates (columns pairs (1, 3) and (2, 4)); \emph{F}-statistics are much smaller; and adjusted-R\textsuperscript{2} do not really pick up a lot of $y$ variation. Compared to the preferred specification in column 6 of table \ref{tab:outcome1}, column 6 here displays an insignificant parameter estimate for the conviction variable. Despite the underwhelming results, there are two interesting interpretations I suggest in this proposal and which I explore further in the final project.

First, the negative effect of electoral infraction, where significant, is extremely small (0.399-0.751 percentage points). On the one hand, this could mean that there are other, more important factors that influence someone's electoral chances. For instance, it could be that opponents are adopting similar strategies thus voters perceive all candidates the same way, regardless of their criminal record \citep{PavaoCorruptionOnlyOption2018}; on the other hand, this could be driven by the fact that I disregard the effect of individual crimes and focus on the aggregated effect of electoral rule-breaking. In either case, these puzzles motive the heterogeneous effects analysis added to the complete paper version. Disaggregating the types of crime both hints at the electoral strategies employed by candidates, since different convictions are likely correlated with different tactics, and uncovers the differential punishment by type of electoral wrongdoing. These are interesting supplemental analyses which will be presented in the dissertation chapter.

The second plausible reading of this result is that I should not be estimating the effect on vote distance using the joint sample of mayor and city councilor candidates, being that majoritarian and proportional elections are very different in nature and so are the candidates running for elected office in each race. The percentage of votes city councilor candidates obtain is much smaller not only as a proportion of the total vote but also with respect to other candidates. Candidates have lower vote shares and are closer to each other in the final vote count. If I split the sample into the different types of candidates, I can test further this hypothesis and check why I see an insignificant effect in table \ref{tab:outcome2}.


\begin{table}[!htbp] \centering
  \caption{The Effect of Electoral Crimes on the Total Vote Share}
  \label{tab:outcome3}
\scriptsize
\begin{tabular}{@{\extracolsep{-2pt}}lD{.}{.}{-3} D{.}{.}{-3} D{.}{.}{-3} D{.}{.}{-3} D{.}{.}{-3} D{.}{.}{-3} }
\\[-1.8ex]\hline
\hline \\[-1.8ex]
                     & \multicolumn{6}{c}{Outcome: Total Vote Share (in percent)} \\
\cline{2-7} \\[-1.8ex]
                     & \multicolumn{1}{c}{OLS} & \multicolumn{1}{c}{OLS} & \multicolumn{1}{c}{Reduced-form} & \multicolumn{1}{c}{Reduced-form} & \multicolumn{1}{c}{IV} & \multicolumn{1}{c}{IV} \\
\\[-1.8ex]           & \multicolumn{1}{c}{(1)} & \multicolumn{1}{c}{(2)} & \multicolumn{1}{c}{(3)} & \multicolumn{1}{c}{(4)} & \multicolumn{1}{c}{(5)} & \multicolumn{1}{c}{(6)}\\
\hline \\[-1.8ex]
 Convicted at Trial  & -12.935^{***} & -10.629^{***} &               &               & -16.795^{***} & -17.865^{***} \\
                     & (.418)        & (.396)        &               &               & (.478)        & (.479) \\
                     &               &               &               &               &               & \\
 Convicted on Appeal &               &               & -12.924^{***} & -11.117^{***} &               &  \\
                     &               &               & (.364)        & (.339)        &               &  \\
                     &               &               &               &               &               & \\
\hline \\[-1.8ex]
Individual Controls  & \multicolumn{1}{c}{-}              & \multicolumn{1}{c}{Yes}             & \multicolumn{1}{c}{-}              & \multicolumn{1}{c}{Yes}             & \multicolumn{1}{c}{-}      & \multicolumn{1}{c}{Yes} \\
\hline \\[-1.8ex]
Observations         & \multicolumn{1}{c}{9,441}          & \multicolumn{1}{c}{9,441}           & \multicolumn{1}{c}{9,441}          & \multicolumn{1}{c}{9,441}           & \multicolumn{1}{c}{9,441}  & \multicolumn{1}{c}{9,441} \\
R$^{2}$              & \multicolumn{1}{c}{.119}           & \multicolumn{1}{c}{.237}            & \multicolumn{1}{c}{.128}           & \multicolumn{1}{c}{.253}            & \multicolumn{1}{c}{.109}   & \multicolumn{1}{c}{.102} \\
Adjusted R$^{2}$     & \multicolumn{1}{c}{.119}           & \multicolumn{1}{c}{.236}            & \multicolumn{1}{c}{.128}           & \multicolumn{1}{c}{.252}            & \multicolumn{1}{c}{.108}   & \multicolumn{1}{c}{.102} \\
Residual Std. Error  & \multicolumn{1}{c}{16.879}         & \multicolumn{1}{c}{15.721}          & \multicolumn{1}{c}{16.790}         & \multicolumn{1}{c}{15.558}          & \multicolumn{1}{c}{16.980} & \multicolumn{1}{c}{17.044} \\
F-Statistic          & \multicolumn{1}{c}{1,277$^{***}$}  & \multicolumn{1}{c}{183$^{***}$}     & \multicolumn{1}{c}{1,390$^{***}$}  & \multicolumn{1}{c}{199$^{***}$}     & \multicolumn{1}{c}{-}      & \multicolumn{1}{c}{-}   \\
                     & \multicolumn{1}{c}{(df = 1; 9439)} & \multicolumn{1}{c}{(df = 16; 9424)} & \multicolumn{1}{c}{(df = 1; 9439)} & \multicolumn{1}{c}{(df = 16; 9424)} & & \\

\hline
\hline \\[-1.8ex]
\multicolumn{7}{p{.98\textwidth}}{\textit{Note:} The regressions here estimate the relationship between being convicted at trial or on appeal, for all candidates who have had their candidacy challenged under charges of electoral irregularities, on their total vote share. Columns 1, 3, and 5 display models not including individual candidate characteristics; columns 2, 4, and 6 include age, gender, marital status, education level, political experience, and the amount spent in their campaign. I estimate robust standard errors for all specifications in this table. $^{*}$p$<$0.1; $^{**}$p$<$0.05; $^{***}$p$<$0.01} \\
\end{tabular}
\end{table}


This preliminary analysis ends with table \ref{tab:outcome3}, where I report the effect of convictions on the total vote share obtained by each candidate in their electoral race. In a similar fashion to table \ref{tab:outcome1}, we see similar magnitudes of parameters across OLS and reduced-form specifications; these parameters also remain statistically significant at 1 percent across all columns. Being convicted of a electoral crime reduces their total vote share by 10.6 and 11.1 percentage points when including candidate characteristics in the regressions. The IV parameter is even larger. Candidates who do not meet all electoral requirements when running for municipal office see their total vote share reduce by 17.9 percentage points, according to the estimate in column 6. This result is consistent with the punishment effect displayed in table \ref{tab:outcome1} and support the overall negative effect of convictions on performance.

\subsection{Further Development} \label{subsec:conclusion_paper1}

This project aims at uncovering the effect of illegal electoral actions on performance. Supplemental to existing literature looking at severe electoral fraud and electoral malfeasance in non or partially democratic regimes, I provide evidence on less known, less understood electoral practices that can also shape the results of elections. I measure such effect in Brazil, one of the largest and highest quality electoral democracies in world at the time of the study. Further contributions of this paper are the use of court documents as data and causal identification using the institutional design of electoral judicial review in Brazil.

Though this is a partial version of the complete paper, I believe I have provided sufficient evidence that this is a promising project. I find substantial and significant negative effects of electoral violation on ballot performance. Being convicted of an electoral crime reduces the probability of election for mayor and city councilor candidates in Brazil by 28.8 percentage points. Beyond the initial analysis, I have laid out the remaining necessary steps for the completion of this project, i.e.~downloading the judicial sentences for each candidacy in the sample, extracting judge information, providing additional tests of instrument strength, and offering heterogeneous treatment effects by electoral crime. In sum, this makes for a strong dissertation chapter.

Finally, this research question is relevant for multiple policy discussions. I offer additional evidence claiming the existence of a negative relationship between crime and electoral performance beyond just corruption. Knowing that voters punish bad behavior, skilled politicians and policymakers can increase monitoring, detection, and prosecution of electoral crimes as a means of weeding out low-quality office-seeking candidates. By looking at heterogeneous effects of electoral infractions, I also produce evidence justifying proportional punishment following each crime's severity. High electoral punishment infractions can be matched by equivalent legal punishment at other instances of the judicial system in Brazil. A final implication of this project is, indeed, an honest discussion on the effectiveness of electoral oversight authorities in the first place. While on the one hand they might prevent low-quality candidates from running, and eventually, being elected, they might simultaneously create barriers to entry that are detrimental to political competition and to the democratic process in developing countries.

\clearpage

% \section{Judicial Favoritism of Politicians: Evidence from Small Claim Courts} \label{sec:paper2}

% \subsection{Introduction} \label{subsec:introduction_paper2}

% \subsection{Institutional Background} \label{subsec:background_paper2}

% \subsection{Data} \label{subsec:data_paper2}

% \subsection{Theory} \label{subsec:theory_paper2}

% \subsection{Empirical Strategy} \label{subsec:methods_paper2}

% \subsection{Preliminary Results} \label{subsec:results_paper2}

% \subsection{Further Development} \label{subsec:conclusion_paper2}

% \clearpage

% \section{Active and Passive Transparency in Brazilian Municipalities} \label{sec:paper3}

% \subsection{Introduction} \label{subsec:introduction_paper3}

% \subsection{Institutional Background} \label{subsec:background_paper3}

% \subsection{Data} \label{subsec:data_paper3}

% \subsection{Theory} \label{subsec:theory_paper3}

% \subsection{Empirical Strategy} \label{subsec:methods_paper3}

% \subsection{Preliminary Results} \label{subsec:results_paper3}

% \subsection{Further Development} \label{subsec:conclusion_paper3}

% \clearpage

\setlength\bibsep{0pt}
\bibliographystyle{apalike}
\bibliography{/Users/aassumpcao/library.bib}

\end{document}