\documentclass[]{article}
\usepackage{lmodern}
\usepackage{amssymb,amsmath}
\usepackage{ifxetex,ifluatex}
\usepackage{fixltx2e} % provides \textsubscript
\ifnum 0\ifxetex 1\fi\ifluatex 1\fi=0 % if pdftex
  \usepackage[T1]{fontenc}
  \usepackage[utf8]{inputenc}
\else % if luatex or xelatex
  \ifxetex
    \usepackage{mathspec}
  \else
    \usepackage{fontspec}
  \fi
  \defaultfontfeatures{Ligatures=TeX,Scale=MatchLowercase}
\fi
% use upquote if available, for straight quotes in verbatim environments
\IfFileExists{upquote.sty}{\usepackage{upquote}}{}
% use microtype if available
\IfFileExists{microtype.sty}{%
\usepackage{microtype}
\UseMicrotypeSet[protrusion]{basicmath} % disable protrusion for tt fonts
}{}
\usepackage[margin=1in]{geometry}
\usepackage{hyperref}
\hypersetup{unicode=true,
            pdfborder={0 0 0},
            breaklinks=true}
\urlstyle{same}  % don't use monospace font for urls
\usepackage{longtable,booktabs}
\usepackage{graphicx,grffile}
\makeatletter
\def\maxwidth{\ifdim\Gin@nat@width>\linewidth\linewidth\else\Gin@nat@width\fi}
\def\maxheight{\ifdim\Gin@nat@height>\textheight\textheight\else\Gin@nat@height\fi}
\makeatother
% Scale images if necessary, so that they will not overflow the page
% margins by default, and it is still possible to overwrite the defaults
% using explicit options in \includegraphics[width, height, ...]{}
\setkeys{Gin}{width=\maxwidth,height=\maxheight,keepaspectratio}
\IfFileExists{parskip.sty}{%
\usepackage{parskip}
}{% else
\setlength{\parindent}{0pt}
\setlength{\parskip}{6pt plus 2pt minus 1pt}
}
\setlength{\emergencystretch}{3em}  % prevent overfull lines
\providecommand{\tightlist}{%
  \setlength{\itemsep}{0pt}\setlength{\parskip}{0pt}}
\setcounter{secnumdepth}{5}
% Redefines (sub)paragraphs to behave more like sections
\ifx\paragraph\undefined\else
\let\oldparagraph\paragraph
\renewcommand{\paragraph}[1]{\oldparagraph{#1}\mbox{}}
\fi
\ifx\subparagraph\undefined\else
\let\oldsubparagraph\subparagraph
\renewcommand{\subparagraph}[1]{\oldsubparagraph{#1}\mbox{}}
\fi

%%% Use protect on footnotes to avoid problems with footnotes in titles
\let\rmarkdownfootnote\footnote%
\def\footnote{\protect\rmarkdownfootnote}

%%% Change title format to be more compact
\usepackage{titling}

% Create subtitle command for use in maketitle
\newcommand{\subtitle}[1]{
  \posttitle{
    \begin{center}\large#1\end{center}
    }
}

\setlength{\droptitle}{-2em}

  \title{}
    \pretitle{\vspace{\droptitle}}
  \posttitle{}
    \author{}
    \preauthor{}\postauthor{}
    \date{}
    \predate{}\postdate{}
  
\usepackage{float}
\usepackage{tikz}
\usepackage{setspace}

\begin{document}

\hypertarget{title1}{%
\section{Electoral Crimes under Democratic Rule: Evidence from
Brazil}\label{title1}}

\hypertarget{summary}{%
\subsection{Summary}\label{summary}}

Office-seeking politicians employ various tactics to get elected,
including actions banned by electoral oversight agencies. Using a unique
dataset on candidate eligibility constructed from judicial decisions of
the Brazilian Electoral Court (TSE), I estimate the impact of electoral
code violations on performance at municipal elections between 2004-2016.
I recover causal effects of violations using a unique feature from
Brazil's electoral system in which candidates run for office, and can be
voted on election day, even if the Electoral Court has not issued a
final ruling on whether their candidacies met all code requirements.

\hypertarget{main-research-question}{%
\subsection{Main Research Question}\label{main-research-question}}

Do politicians who violate electoral code perform better at the ballot
than those who do not?

\hypertarget{hypotheses}{%
\subsection{Hypotheses}\label{hypotheses}}

\begin{enumerate}
\item
  Candidates who are cleared of electoral violations will perform better
  at municipal elections than candidates who are found guilty.
\item
  Conviction-free candidates will performance proportionately better
  when there are more convicted politicians in the same race.
\item
  The electoral punishment for electoral infractions is increasing in
  their severity (e.g.~corruption, abuse of power, vote buying,
  unauthorized marketing, etc).
\end{enumerate}

\hypertarget{outcomes}{%
\subsection{Outcomes}\label{outcomes}}

\begin{enumerate}
\item
  Whether candidate was the most voted for in mayor elections
  (majoritarian system) or voted in for elected city council seats
  (proportional system) -- equivalent to being elected had they all been
  cleared of accusations.
\item
  Vote share.
\item
  Vote distance to elected candidates.
\end{enumerate}

\hypertarget{identification-strategy}{%
\subsection{Identification Strategy}\label{identification-strategy}}

IV where the instrumented (explanatory) variable is the candidacy ruling
issued by the Electoral Court before the elections (trial stage) and the
instrument is the decision issued after the elections (appeals stage)
for all politicians whose candidacy authorization had been appealed but
had not yet been tried by the appellate panel on election day.

\hypertarget{data}{%
\subsection{Data}\label{data}}

Election results, candidate, judge, and municipal controls for municipal
elections in Brazil between 2004-2016. All available at various
Electoral Court (TSE) websites and the National Statistics Office
(IBGE).

\hypertarget{contribution-to-literature}{%
\subsection{Contribution to
Literature}\label{contribution-to-literature}}

It is the first paper to measure the electoral punishment for electoral
crimes under democratic regimes; it implements a new identification
strategy using different stages of the judicial process; it uses a novel
database of primary data on judicial rulings from electoral oversight
authorities.


\end{document}
