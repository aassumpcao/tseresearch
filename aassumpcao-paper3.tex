\documentclass[]{article}
\usepackage{lmodern}
\usepackage{amssymb,amsmath}
\usepackage{ifxetex,ifluatex}
\usepackage{fixltx2e} % provides \textsubscript
\ifnum 0\ifxetex 1\fi\ifluatex 1\fi=0 % if pdftex
  \usepackage[T1]{fontenc}
  \usepackage[utf8]{inputenc}
\else % if luatex or xelatex
  \ifxetex
    \usepackage{mathspec}
  \else
    \usepackage{fontspec}
  \fi
  \defaultfontfeatures{Ligatures=TeX,Scale=MatchLowercase}
\fi
% use upquote if available, for straight quotes in verbatim environments
\IfFileExists{upquote.sty}{\usepackage{upquote}}{}
% use microtype if available
\IfFileExists{microtype.sty}{%
\usepackage{microtype}
\UseMicrotypeSet[protrusion]{basicmath} % disable protrusion for tt fonts
}{}
\usepackage[margin=1in]{geometry}
\usepackage{hyperref}
\hypersetup{unicode=true,
            pdfborder={0 0 0},
            breaklinks=true}
\urlstyle{same}  % don't use monospace font for urls
\usepackage{longtable,booktabs}
\usepackage{graphicx,grffile}
\makeatletter
\def\maxwidth{\ifdim\Gin@nat@width>\linewidth\linewidth\else\Gin@nat@width\fi}
\def\maxheight{\ifdim\Gin@nat@height>\textheight\textheight\else\Gin@nat@height\fi}
\makeatother
% Scale images if necessary, so that they will not overflow the page
% margins by default, and it is still possible to overwrite the defaults
% using explicit options in \includegraphics[width, height, ...]{}
\setkeys{Gin}{width=\maxwidth,height=\maxheight,keepaspectratio}
\IfFileExists{parskip.sty}{%
\usepackage{parskip}
}{% else
\setlength{\parindent}{0pt}
\setlength{\parskip}{6pt plus 2pt minus 1pt}
}
\setlength{\emergencystretch}{3em}  % prevent overfull lines
\providecommand{\tightlist}{%
  \setlength{\itemsep}{0pt}\setlength{\parskip}{0pt}}
\setcounter{secnumdepth}{5}
% Redefines (sub)paragraphs to behave more like sections
\ifx\paragraph\undefined\else
\let\oldparagraph\paragraph
\renewcommand{\paragraph}[1]{\oldparagraph{#1}\mbox{}}
\fi
\ifx\subparagraph\undefined\else
\let\oldsubparagraph\subparagraph
\renewcommand{\subparagraph}[1]{\oldsubparagraph{#1}\mbox{}}
\fi

%%% Use protect on footnotes to avoid problems with footnotes in titles
\let\rmarkdownfootnote\footnote%
\def\footnote{\protect\rmarkdownfootnote}

%%% Change title format to be more compact
\usepackage{titling}

% Create subtitle command for use in maketitle
\newcommand{\subtitle}[1]{
  \posttitle{
    \begin{center}\large#1\end{center}
    }
}

\setlength{\droptitle}{-2em}

  \title{}
    \pretitle{\vspace{\droptitle}}
  \posttitle{}
    \author{}
    \preauthor{}\postauthor{}
    \date{}
    \predate{}\postdate{}
  
\usepackage{float}
\usepackage{tikz}
\usepackage{setspace}

\begin{document}

\hypertarget{title3}{%
\section{Active and Passive Transparency in Brazilian
Municipalities}\label{title3}}

\hypertarget{summary}{%
\subsection{Summary}\label{summary}}

An important part of government accountability is the obligation of
public officials to inform and explain their actions
{[}@SchedlerConceptualizingAccountability2012;
@BovensAnalysingAssessingAccountability2007{]}. In this paper, I propose
and analyze two related forms of government accountability: \emph{active
transparency}, in which government actively reveals policy information
via intra-government auditing and monitoring, and \emph{passive
transparency}, in which government passively reveals information through
freedom of information requests. Using a natural two-by-two factorial
experiment design in Brazilian municipalities between 2006 and 2017, I
measure the effects of active and passive transparency on government
performance, sanctions, corruption, and transparency outcomes.

\hypertarget{main-research-question}{%
\subsection{Main Research Question}\label{main-research-question}}

Do passive transparency measures contribute anything more than active
transparency to improve government performance and increase the number
of sanctions applied for government wrongdoing?

\hypertarget{hypotheses}{%
\subsection{Hypotheses}\label{hypotheses}}

\begin{enumerate}
\item
  \emph{Active transparency} measures unconditionally improve
  performance and increase the number of individual and company-wide
  sanctions.
\item
  \emph{Passive transparency} only marginally improves performance and
  increases sanctions when \emph{active transparency} policies are in
  place.
\item
  In the absence of \emph{active transparency} measures, \emph{passive
  transparency} has no effect on improving performance and does not
  increase the number of sanctions for individuals and companies found
  guilty of any wrongdoing.
\end{enumerate}

\hypertarget{outcomes}{%
\subsection{Outcomes}\label{outcomes}}

\begin{enumerate}
\item
  Performance (across all groups):

  \begin{enumerate}
  \item
    number of online or services available to the public.
  \item
    the existence of municipal development plan.
  \end{enumerate}
\item
  Sanctions (across all groups):

  \begin{enumerate}
  \item
    whether municipality had any public official convicted/fired for
    wrongdoing.
  \item
    whether local companies have been entered into blacklist of
    government providers.
  \item
    whether municipality was targeted by Federal Police in corruption
    crackdowns.
  \end{enumerate}
\item
  Corruption (across passive transparency groups):

  \begin{enumerate}
  \item
    corruption findings over total investigations.
  \item
    amount potentially lost to corruption over total amount
    investigated.
  \end{enumerate}
\item
  Transparency (across active transparency groups):

  \begin{enumerate}
  \item
    whether municipality responded in time to four FOIA requests.
  \item
    whether municipality provided correct answers to four FOIA requests.
  \end{enumerate}
\end{enumerate}

\hypertarget{identification-strategy}{%
\subsection{Identification Strategy}\label{identification-strategy}}

Natural experiment coming from the combination of two simultaneous
exogenous shocks: randomized audits (active transparency) plus the
nationwide implementation of the freedom of information act (FOIA) in
2012 (passive transparency). Municipalities fall into one of three
treatments or one control group: audits after FOIA (active and passive
transparency), audits before FOIA (active transparency), non-audit after
FOIA (passive transparency), and non-audit before FOIA (control).

\hypertarget{data}{%
\subsection{Data}\label{data}}

Socioeconomic factors and policy outcomes from the National Statistics
Office (IBGE); Random audits and transparency measures from two programs
run by the Office of the Comptroller-General (CGU); Sanctions for
individuals and companies and crackdowns from CGU; Convictions from the
National Council of Justice (CNJ).

\hypertarget{contribution-and-literature}{%
\subsection{Contribution and
Literature}\label{contribution-and-literature}}

First paper providing disaggregated evidence for the effect of passive
transparency (FOIA) in development settings; paper advances theory by
breaking transparency into active and passive arms; new transparency
dataset and ingenious research design.


\end{document}
