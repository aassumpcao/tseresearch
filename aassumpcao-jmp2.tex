\documentclass[]{article}
\usepackage{lmodern}
\usepackage{amssymb,amsmath}
\usepackage{ifxetex,ifluatex}
\usepackage{fixltx2e} % provides \textsubscript
\ifnum 0\ifxetex 1\fi\ifluatex 1\fi=0 % if pdftex
  \usepackage[T1]{fontenc}
  \usepackage[utf8]{inputenc}
\else % if luatex or xelatex
  \ifxetex
    \usepackage{mathspec}
  \else
    \usepackage{fontspec}
  \fi
  \defaultfontfeatures{Ligatures=TeX,Scale=MatchLowercase}
\fi
% use upquote if available, for straight quotes in verbatim environments
\IfFileExists{upquote.sty}{\usepackage{upquote}}{}
% use microtype if available
\IfFileExists{microtype.sty}{%
\usepackage{microtype}
\UseMicrotypeSet[protrusion]{basicmath} % disable protrusion for tt fonts
}{}
\usepackage[margin=1in]{geometry}
\usepackage{hyperref}
\hypersetup{unicode=true,
            pdfborder={0 0 0},
            breaklinks=true}
\urlstyle{same}  % don't use monospace font for urls
\usepackage{longtable,booktabs}
\usepackage{graphicx,grffile}
\makeatletter
\def\maxwidth{\ifdim\Gin@nat@width>\linewidth\linewidth\else\Gin@nat@width\fi}
\def\maxheight{\ifdim\Gin@nat@height>\textheight\textheight\else\Gin@nat@height\fi}
\makeatother
% Scale images if necessary, so that they will not overflow the page
% margins by default, and it is still possible to overwrite the defaults
% using explicit options in \includegraphics[width, height, ...]{}
\setkeys{Gin}{width=\maxwidth,height=\maxheight,keepaspectratio}
\IfFileExists{parskip.sty}{%
\usepackage{parskip}
}{% else
\setlength{\parindent}{0pt}
\setlength{\parskip}{6pt plus 2pt minus 1pt}
}
\setlength{\emergencystretch}{3em}  % prevent overfull lines
\providecommand{\tightlist}{%
  \setlength{\itemsep}{0pt}\setlength{\parskip}{0pt}}
\setcounter{secnumdepth}{5}
% Redefines (sub)paragraphs to behave more like sections
\ifx\paragraph\undefined\else
\let\oldparagraph\paragraph
\renewcommand{\paragraph}[1]{\oldparagraph{#1}\mbox{}}
\fi
\ifx\subparagraph\undefined\else
\let\oldsubparagraph\subparagraph
\renewcommand{\subparagraph}[1]{\oldsubparagraph{#1}\mbox{}}
\fi

%%% Use protect on footnotes to avoid problems with footnotes in titles
\let\rmarkdownfootnote\footnote%
\def\footnote{\protect\rmarkdownfootnote}

%%% Change title format to be more compact
\usepackage{titling}

% Create subtitle command for use in maketitle
\newcommand{\subtitle}[1]{
  \posttitle{
    \begin{center}\large#1\end{center}
    }
}

\setlength{\droptitle}{-2em}

  \title{}
    \pretitle{\vspace{\droptitle}}
  \posttitle{}
    \author{}
    \preauthor{}\postauthor{}
    \date{}
    \predate{}\postdate{}
  
\usepackage{float}
\usepackage{tikz}
\usepackage{setspace}

\begin{document}

\hypertarget{title}{%
\section{Judicial Favoritism of Politicians: Evidence from Small Claim
Courts}\label{title}}

Andre Assumpcao

\hypertarget{summary}{%
\subsection{Summary}\label{summary}}

Judicial favoritism has long been a subject of research in law,
economics, and political science. However, scholars have mainly focused
on the effect of gender and ethnicity on rulings but have largely
ignored whether judges treat politicians the same way as ordinary
citizens. I use a unique dataset of judicial decisions in small claims
courts in the state of São Paulo, Brazil, to answer this question, where
cases are assigned to judges at random, to verify whether local
politicians have a higher winning rate against plaintiffs or defendants.
I combine empirical strategies in Shayo and Zussman (2011), Abrams,
Bertrand, and Mullainathan (2012), and Sanchez-Martinez (2018) to test
random assignment and provide robustness checks against potential
spurious relationships between being a politician and having a favorable
court outcome.

\hypertarget{main-research-question}{%
\subsection{Main research Question}\label{main-research-question}}

Are individuals running for office more likely to receive favorable
rulings in small claim courts?

\hypertarget{hypotheses}{%
\subsection{Hypotheses}\label{hypotheses}}

\begin{enumerate}
\item
  Politicians have a higher winning rate at the trial stage in small
  court claims against their counterparts.
\item
  Proximity to elections increases the winning rates for politicians on
  the campaign trail.
\end{enumerate}

\hypertarget{outcomes}{%
\subsection{Outcomes}\label{outcomes}}

\begin{enumerate}
\item
  Whether politicians have had the case ruled in their favor.
\item
  The amount awarded to (or avoided by) politicians in court cases.
\end{enumerate}

\hypertarget{identification-strategy}{%
\subsection{Identification Strategy}\label{identification-strategy}}

Natural experiment. State Courts assign cases at random when the
judicial district has more than one judge on the bench. Abrams,
Bertrand, and Mullainathan (2012) suggest several tests of random
assignment which I use here.

\hypertarget{data}{%
\subsection{Data}\label{data}}

São Paulo State Court (TJ-SP) rulings involving candidates running for
office in the State of São Paulo in the 2010, 2012, 2014, and 2016
electoral cycles. Judicial district, judge and politicians' individual
characteristics from the Electoral Court (TSE), TJ-SP, and the National
Statistics Office (IBGE).

\hypertarget{contribution-and-literature}{%
\subsection{Contribution and
Literature}\label{contribution-and-literature}}

It is the first paper to investigate judicial bias for individual
politicians and it contributes to the literature on the benefits of
political connectedness.


\end{document}
